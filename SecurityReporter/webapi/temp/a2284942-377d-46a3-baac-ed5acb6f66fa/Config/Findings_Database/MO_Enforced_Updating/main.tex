%
%
%		Finding: Template
%		Author: the DR
%
%
\renewcommand{\FindingAuthor}{Michal Olencin}
% DO NOT USE \par, \newline or any other line breaking command in FindingName => report will not build
\renewcommand{\FindingName}{Missing Enforced Updating}
\renewcommand{\Location}{dummyapplication.apk, dummyapplication.ipa}
\renewcommand{\Component}{Multiple}
\renewcommand{\FoundWith}{Manual Testing}
\renewcommand{\TestMethod}{Manual}
\renewcommand{\CVSS}{N/A}
\renewcommand{\CVSSvector}{N/A}
\renewcommand{\CWE}{691}
% Poor-man's combo boxes:
% High, Medium, Low, Info, TBR (To Be Rated)
\renewcommand{\Criticality}{Info}
% Easy, Average, Hard, TBR (To Be Rated)
\renewcommand{\Exploitability}{Hard}
% Access control, Application Design, Information Disclosure, Outdated Software, Security Configuration
\renewcommand{\Category}{Outdated Software}
% Easy, Average, Difficult, TBR (To Be Rated)
\renewcommand{\Detectability}{Easy}


\ReportFindingHeader{\FindingName}


%-------------------------------------------
%	Details                                |
%-------------------------------------------

\subsection*{Details}

The both applications, Android ans iOS, are missing enforced updating, which assures running updated and fully patched application.


%-<Details>
%-------------------------------------------
%	Impact                                 |
%-------------------------------------------

\subsection*{Impact}

The missing enforced updating poses the risk of exploitation of known vulnerabilities, increases the attack surface of outdated app, and may result in compliance violations.

%-<Impact>
%-------------------------------------------
%	Repeatability                          |
%-------------------------------------------

\subsection*{Repeatability}

Try to change version of the application to an older version.
Rebuild the application and run it.
User is able to interact with the older version of the application.

 
%-<Repeatability>
%-------------------------------------------
%	Countermeasures                        |
%-------------------------------------------

\subsection*{Countermeasures}

When the app is opened, check whether any new updates have been released for the application.
If the app is outdated, do not allow the user to interact with the application until it is updated.

To check for new updates, the \texttt{\href{https://developer.android.com/reference/com/google/android/play/core/appupdate/AppUpdateManager}{AppUpdateManager}} for Android applications can be used.

For iOS applications, the \texttt{http://itunes.apple.com/lookup?id=<BundleId>} API call can be used.

%-<Countermeasures>
%-------------------------------------------
%	References - pulls bib entries         |
%-------------------------------------------
\pagebreak
\subsection*{References}

This finding references the following information sources:

\begin{itemize}
    \item \href{https://developer.android.com/reference/com/google/android/play/core/appupdate/AppUpdateManager}{AppUpdateManager Documentation}
    \item \href{https://performance-partners.apple.com/search-api}{iTunes Search API}
	\item \bibentry{CWE-691}
\end{itemize}


%-<References>
