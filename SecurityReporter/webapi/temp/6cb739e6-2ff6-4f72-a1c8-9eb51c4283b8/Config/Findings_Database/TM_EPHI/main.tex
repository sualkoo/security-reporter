%
%
%		Finding: Template
%		Author: the DR
%
%
\renewcommand{\FindingAuthor}{Taksh Medhavi}
% DO NOT USE \par, \newline or any other line breaking command in FindingName => report will not build
\renewcommand{\FindingName}{ePHI is stored on device without encryption}
\renewcommand{\Location}{dummyapplication.apk}
\renewcommand{\Component}{Android local storage}
\renewcommand{\FoundWith}{TODOTODO}
\renewcommand{\TestMethod}{TODOTODO}
\renewcommand{\CVSS}{8.0}
\renewcommand{\CVSSvector}{CVSS:3.1/AV:A/AC:L/PR:L/UI:N/S:U/C:H/I:H/A:H}
\renewcommand{\CWE}{359}
% Poor-man's combo boxes:
% High, Medium, Low, Info, TBR (To Be Rated)
\renewcommand{\Criticality}{High}
% Easy, Average, Hard, TBR (To Be Rated)
\renewcommand{\Exploitability}{Easy}
% Access control, Application Design, Information Disclosure, Outdated Software, Security Configuration
\renewcommand{\Category}{Undefined}
% Easy, Average, Difficult, TBR (To Be Rated)
\renewcommand{\Detectability}{Easy}


\ReportFindingHeader{\FindingName}


%-------------------------------------------
%	Details                                |
%-------------------------------------------

\subsection*{Details}

Application allows user to save ePHI (Electronic Protected Health Information) on device and patient information is stored in plain text in HTML file format. Application stores HTML file in \texttt{android > data > com.siemenshealthineers.dummyapplicationapp > files} directory. The file contains sensitive information such as patient vitals, allergies, diagnostic results and medication in it.


%-<Details>
%-------------------------------------------
%	Impact                                 |
%-------------------------------------------



\subsection*{Impact}

Third party application installed in mobile devices can access ePHI stored in application data directory and since data at rest is stored without encryption, attacker can read contents of file which can lead to loss of confidentiality and violation of healthcare compliance. Application with external storage read/write permission can affect file integrity, as well as, application availability. In another scenario, attacker with local access of device can use file manager to access patient data. 

%-<Impact>
%-------------------------------------------
%	Repeatability                          |
%-------------------------------------------

\newpage


\subsection*{Repeatability}

User can download patient summary details in device in plaintext HTML file as shown in \cref{figure:003.ephi_2.jpg}. \cref{figure:005.ephi_in_html_file} shows that HTML file contains ePHI such as patient vitals, allergies, diagnostic results and medication in it.

\begin{figure}[h]
	\begin{subfigure}{0.5\textwidth}
	\includegraphics[width=0.9\linewidth]{\CurrentFilePath/ProperScreenshot.png} 
	\caption{Downloading patient information}
	\label{figure:003.ephi_2.jpg}
	\end{subfigure}
	\begin{subfigure}{0.5\textwidth}
	\includegraphics[width=0.9\linewidth]{\CurrentFilePath/ProperScreenshot.png}
	\caption{ePHI in plaintext HTML file}
	\label{figure:005.ephi_in_html_file}
	\end{subfigure}
	\caption{ePHI without data at rest encryption}
	\label{fig:005.ephi_in_html_file}
\end{figure}

\newpage

User can also open patient summary HTML file using file manager and access ePHI as shown in \cref{figure:006.patient_data_file_location}.

\begin{figure}[H]
\centering
\includegraphics[scale=1.0,frame]{\CurrentFilePath/ProperScreenshot.png}
\caption{EPHI access via file manager}
\label{figure:006.patient_data_file_location}
\end{figure}




 
%-<Repeatability>
%-------------------------------------------
%	Countermeasures                        |
%-------------------------------------------



\subsection*{Countermeasures}

Application allows ePHI to be downloaded in plain HTML without encryption at rest. Application should handle patient data as per healthcare compliances applicable and should provide PDF report with password.


%-<Countermeasures>
%-------------------------------------------
%	References - pulls bib entries         |
%-------------------------------------------



\subsection*{References}

This finding references the following information sources:

\begin{itemize}
	\item \href{https://www.first.org/cvss/calculator/3.0#CVSS:3.0/AV:A/AC:L/PR:L/UI:N/S:U/C:H/I:H/A:H}
	{CVSS 8.0}
	% \item \href{https://cwe.mitre.org/data/definitions/359.html}
	% {CWE-359}
	\item \bibentry{CWE-359}
\end{itemize}

%-<References>



