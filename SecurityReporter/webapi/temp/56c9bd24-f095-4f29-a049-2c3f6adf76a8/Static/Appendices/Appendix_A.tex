%
%		File:		Appendix_A.tex
%		Author: 	Dusan Repel (SHS TE DC CYS CSA)
%		Date:		02/09/2020
%
%
\clearpage

\chapter{Appendix A}
\label{appendix:AppendixA}


%----------------------------------------------------------------------------------------
%	CRITICALITY LEVELS
%----------------------------------------------------------------------------------------
\section{Criticality Levels}
\label{appendix:CriticalityLevels} 

Each finding in the \ReportProjectType is associated with a certain criticality level. The criticality level is an estimation of the potential impact of a finding and the likelihood of its exploitation. Furthermore, laws on data privacy, Siemens Information Security Regulations~\cite{InfoSecPolicy} and information security best practices are incorporated into the rating process.

The following criticality levels are used in this \ReportProjectType:


\begin{xltabular}{\textwidth}{|l|c|X|}
	\hline
	\cellcolor{grey230} \textbf{Severity} & \cellcolor{grey230} \textbf{CVSS Score} & \cellcolor{grey230} \textbf{Description} \\
	\hline
	\cellcolor{dark-red-shs} \color{black} \textbf{Critical} & 9.0 - 10.0 & Critical findings can be readily compromised with publicly available malware or exploits. \\
	\hline
	\cellcolor{red-shs} \color{black} \textbf{High} & 7.0 - 8.9 & Vulnerabilities that can lead to worst case scenarios in a very direct way without too complex preconditions, e.g. a missing security patch that allows taking over an operating system, an SQL Injection that allows direct access to the database or a privilege escalation to an administrative account. \\
	\hline
	\cellcolor{orange-shs} \color{black} \textbf{Medium} & 4.0 - 6.9 & Vulnerabilities that might trigger worst case scenarios in an indirect way or that might only work under certain circumstances, e.g. a cross-side request- forgery-attack (CSRF) where an attacker needs to send an email to a certain person, this person needs to click on a link in this email and this person already needs to be logged into a certain application to trigger a command execution in an application (phishing attacks). \\
	\hline
	\cellcolor{yellow-shs}\color{black} \textbf{Low} & 0.1 - 3.9 & Vulnerabilities that are neither directly, nor indirectly exploitable but increase the likelihood or impact of another vulnerability. Additionally, vulnerabilities are classified low if they require highly advanced hacking skills or very complex preconditions and, hence, the likelihood of exploitation is extremely low, or the impact is minimal. Examples are error message revealing software version numbers or internal path information. \\
	\hline
	\cellcolor{grey-shs}\color{black} \textbf{Information} & N/A & Additional observations or notes from a security perspective: observations of problematic behavior for which no clear evidence could be found, or which do not pose a direct security risk, but should still be reviewed by the asset owner. \\
	\hline
	\caption{Criticality Levels} \label{table:CriticalityLevels}
\end{xltabular}


%----------------------------------------------------------------------------------------
%	OVERALL THREAT EXPOSURE
%----------------------------------------------------------------------------------------
\section{Overall Threat Exposure}
\label{appendix:OverallThreatExposure}
The \textbf{“Overall Threat Exposure”} determines the current security state of the \PrintAssetName. The overall criticality is based on two key factors: 
\begin{itemize}
	\item	Criticality of the findings
	\item	Exploitation of defined worst-case scenarios
\end{itemize}

The following triggers determine the criticality of the “Overall Threat Exposure”:


\begin{xltabular}{\textwidth}{|l|X|}
	\hline
	\cellcolor{grey230} \textbf{Threat Exposure} & \cellcolor{grey230} \textbf{Trigger}\\
	\hline
	\cellcolor{dark-red-shs} \textbf{Critical} & At least one \textbf{Critical} finding. \\
	\hline
	\cellcolor{red-shs} \textbf{High} & At least one \textbf{High} finding \textit{OR/AND} \newline At least one worst-case scenario triggered by a \textbf{High} or \textbf{Medium} finding. \\
	\cellcolor{orange-shs} \textbf{Medium} & No High findings. \newline No worst-case scenario triggered. \newline At least one Medium finding. \\
	\hline
	\cellcolor{yellow-shs} \textbf{Low} & No High or Medium findings. \newline No worst-case scenarios triggered. \newline Only Low findings. \\
	\hline
\caption{Overall Threat Exposure} \label{table:OverallThreatExposureAppendix}
\end{xltabular}
	
In the case of a worst-case scenario being triggered by a \textbf{Low} finding and no other findings higher than \textbf{Low} being found, the overall threat exposure might be rated higher than \textbf{Low}. This will be determined on a case-by-case basis. 


%----------------------------------------------------------------------------------------
%	TESTING APPROACHES
%----------------------------------------------------------------------------------------
\section{Testing Approaches}
\label{appendix:TestingApproaches}

\paragraph{Manual Testing}	Manual testing is done by professional security experts. It requires collection of security data about the \PrintAssetName manually in order to identify vulnerabilities and exploit them.

\paragraph{Automatic Scans}	Automatic scans are done by in-house and third-party tools and are much faster than manual testing. The testing software performs automatic actions, which would otherwise be very time consuming for a security expert. The scan then generates a report with information about every finding that was found during the scan.  

\paragraph{Manual Verification}	Automatic scans produce a lot of results with many false positives/false negatives. Therefore, a manual check by a professional security expert is required to verify the results and eliminate all false positives/negatives.


%----------------------------------------------------------------------------------------
%	TEST PROTOCOL
%----------------------------------------------------------------------------------------
\clearpage
\section{Test Protocol}
\textbf{Target:} \TargetTestProtocol \newline

\begin{xltabular}{\linewidth}{|l|X|l|l|}
	\cellcolor{grey230} \textbf{OWASP Control} & \cellcolor{grey230} \textbf{OWASP Testing Method} & \cellcolor{grey230} \textbf{Result} & \cellcolor{grey230} \textbf{Comment}\\
	\csvreader[separator=semicolon,late after line=\\\midrule,late after last line=\\\bottomrule]
	  {./Config/Test_Protocol/owasp.csv}
	  {}
	  {\csvcoli & \csvcolii & \csvcoliii & \csvcoliv}
\caption{OWASP Testing Guide v4} \label{tab:OWASPTestingGuidev4} \\
\end{xltabular}


%----------------------------------------------------------------------------------------
%	DEFINITIONS AND ABBREVIATIONS
%----------------------------------------------------------------------------------------
% \section{Definitions and Abbreviations}
% \label{appendix:DefinitionsAndAbbreviations}

% (insert)